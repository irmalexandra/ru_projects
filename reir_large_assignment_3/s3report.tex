% Skeleton report for Assignment S3, in Reiknirit, fall 2020, at Reykjavik University
%  (c) 2017, Magnus M. Halldorsson
%
% Builds on the assignment Pattern Recognition at Princeton University, (c) Kevin Wayne
%
% Format is based on the following document:
% A basic LaTeX document for a handin with a standard RU title page
%  (c) 2013, Tómas Ken Magnússon, 

% If you want the title to appear on a separate page, change notitlepage to titlepage
\documentclass[11pt,a4paper,notitlepage]{article}
\usepackage[utf8]{inputenc}
\usepackage[T1]{fontenc}
% If your hand-in is in icelandic change english to icelandic
% Note: This has nothing to do with Icelandic characters, they
% can always be used. This just tells other packages what
% language you are using and changes the hyphenation used by LaTeX
% If icelandic is selected2 a shorthand, "` and "', is also included
% for Icelandic quotation marks. They can also obtained by using
% ,, and ``
\usepackage[english]{babel}
\usepackage{amsmath, amsthm, amssymb, amsfonts}
\usepackage{graphicx}
\usepackage{enumerate}
% To use the whole A4-page
% See: ftp://ftp.tex.ac.uk/tex-archive/macros/latex/contrib/geometry/geometry.pdf
% and http://en.wikibooks.org/wiki/LaTeX/Document_Structure
\usepackage{geometry}
% For header and footer
% See: ftp://ctan.tug.org/tex-archive/macros/latex/contrib/ fancyhdr/fancyhdr.pdf
% and http://en.wikibooks.org/wiki/LaTeX/Document_Structure
\usepackage{fancyhdr}
% For prettier tables
% See: http://ctan.mackichan.com/macros/latex/contrib/booktabs/booktabs.pdf
% and  http://en.wikibooks.org/wiki/LaTeX/Tables
%\usepackage{booktabs}
\usepackage{listings} % For code listing

% \usepackage{sagetex}
\usepackage{hyperref}
\usepackage{caption}
\usepackage[usenames,dvipsnames,svgnames,table]{xcolor}
\usepackage{subfig}  % To fit two tables side-by-side

%%%%%%%%%%%%%%%%%%%%%%%%%%%%%%%%%%%%%%%%%%%%%%%%%%%%%%%%%%%%%
%                        Setup
%%%%%%%%%%%%%%%%%%%%%%%%%%%%%%%%%%%%%%%%%%%%%%%%%%%%%%%%%%%%%

% Set the margins of the paper. By default LaTeX uses huge margins
\geometry{includeheadfoot, margin=2.5cm}
% you can also use
% \geometry{a4paper}
% End of margins setup

% Settings for listings
\lstset{language=Java,numbers=left,backgroundcolor=\color{light-gray},
        basicstyle=\scriptsize\ttfamily,frame=single,tabsize=4,
        captionpos=t, numbers=left,
        keywordstyle=\color{javapurple}\bfseries,
        stringstyle=\color{javared},
        commentstyle=\color{javagreen},
        morecomment=[s][\color{javadocblue}]{/**}{*/},}
% End of settings for listings

% Custom colors for listings
\definecolor{light-gray}{gray}{0.95}
\definecolor{javared}{rgb}{0.6,0,0} % for strings
\definecolor{javagreen}{rgb}{0.25,0.5,0.35} % comments
\definecolor{javapurple}{rgb}{0.5,0,0.35} % keywords
\definecolor{javadocblue}{rgb}{0.25,0.35,0.75} % javadoc
% End of custom colors for listings


% Fill in any relevant information
% Leave the fields inside the {} empty if they do not apply
\newcommand{\semester}{Fall 2020}
\newcommand{\coursename}{Reiknirit}
\newcommand{\courseid}{T-301-REIR}
\newcommand{\assignment}{S3: Kd-trees}
\newcommand{\problemtitle}{Problem}
\newcommand{\dateofcompilation}{\today}


% Setup header and footer
% Headers
\pagestyle{fancy} % To get the header and footer
\chead{\small \textsc{\assignment}}
\rhead{\small \textsc{\coursename}}

% Footers
%\lfoot{Left footer text}
%\cfoot{\thepage} % This is the default behaviour
%\rfoot{Right footer text}


% Custom dot for itemize
\renewcommand{\labelitemi}{$\cdot$}

% If you don't want a line below the header or above the footer,
% change the appropriate header/footerrulewidth to 0pt
\setlength{\headheight}{15.2pt} % This is set to avoid a warning
\renewcommand{\headrulewidth}{0.4pt}
\renewcommand{\footrulewidth}{0.4pt}
% End of header and footer setup

% Setup Problem/Solution environments // You probably don't need this
%\theoremstyle{plain}
%\newtheorem{problem}{Dæmi}
%\theoremstyle{remark}
%\newtheorem*{solution}{Lausn}
% End of Problem/Solution environments setup
%\theoremstyle{plain}
%\newtheorem*{proposition}{Proposition}

\DeclareCaptionLabelFormat{andtable}{#1~#2  \&  \tablename~\thetable}

% Custom problem (so you can provide the problem name)
%\newenvironment{cproblem}[1]{\begin{trivlist}
%\item[\hskip \labelsep {\bfseries \problemtitle}\hskip \labelsep {\bfseries#1.}]\begin{itshape}}{\end{itshape}\end{trivlist}}
% End of Problem/Solution environments setup

% The title page
\newcommand{\maketitlepage}[1]
{
    \begin{titlepage}

        \begin{center}
            \includegraphics[width=0.55\textwidth]{./rulogo.png}\\[1.5cm]

            \textsc{\huge \semester}\\[0.8cm]

            {\textsc{\Huge \courseid, \coursename}}\\[0.4cm]
            \textsc{\LARGE }\\[2.5cm]

            \textbf{\textsc{\Huge #1}}\\[3cm]


            \textsc{\LARGE (Name of student 1), (email), (kt)}\\  %%%  CHANGE THIS """
            \textsc{\LARGE (Name of student 2), (email), (kt)}\\  %%%  CHANGE THIS """
            \textsc{\LARGE (Name of student 3), (email), (kt)}\\[0.6cm]  %%%  CHANGE T            \textsc{\huge \students}\\[0.4cm]
            \textsc{\LARGE Group S2-xx}\\[1cm]
            \textsc{\Large \dateofcompilation}


        \end{center}

        \vfill

        % You may also want to add the name of your teaching assistant
        \begin{flushleft}
            \textsc{\Large TA: Eiríkur Fjalar}   %%%  CHANGE THIS """

        \end{flushleft}

    \end{titlepage}
}
\newcommand{\command}[1]{\texttt{\textbackslash{}#1}}

\newcommand{\explanation}[1]{}  %% Use this when turning in the report
%\newcommand{\explanation}[1]{\begin{quote}\emph{#1} \end{quote}}  %% Use this to include directions


%%%%%%%%%%%%%%%%%%%%%%%% END OF SETUP %%%%%%%%%%%%%%%%%%%%%%%%


\begin{document}
% Create the title page
    \maketitlepage{\assignment}

\explanation{Directions on performing the assignment are showed here in italics (like this). These should not be included in the report you submit.}


\section{Implementation}

\subsection*{Node}

\explanation{Describe the Node data type you used to implement the
  2d-tree data structure.}

\subsection*{Range Search}
\explanation{Describe your method for range search in a kd-tree.}

\subsection*{Nearest Neighbor Search}
\explanation{Describe your method for nearest neighbor search in a kd-tree.}


\section{Analysis}

\subsection*{Memory}
\explanation{
   Give the total memory usage in bytes (using tilde notation and 
   the standard 64-bit memory cost model) of your 2d-tree data
   structure as a function of the number of points N. Justify your
   answer below.
% 
   Include the memory for all referenced objects (deep memory),
   including memory for the nodes, points, and rectangles.
}

bytes per Point2D: 32 bytes

bytes per RectHV:

bytes per KdTree of N points (using tilde notation):   $\sim$
% \ (include the memory for any referenced \texttt{Node}, \texttt{Point2D} and \texttt{RectHV} objects)


\subsection*{Running Time}
\explanation{
   Give the expected running time in seconds (using tilde notation)
   to build a 2d-tree on N random points in the unit square.
   Use empirical evidence by creating a table of different values of N
   and the timing results. (Do not count the time to generate the N 
   points or to read them in from standard input.)
}


\paragraph*{Nearest Neighbor}
\explanation{
   How many nearest neighbor calculations can your brute-force
   implementation perform per second for input100K.txt (100,000 points)
   and input1M.txt (1 million points), where the query points are
   random points in the unit square? Explain how you determined the
   operations per second. (Do not count the time to read in the points
   or to build the 2d-tree.) 
(The files are in \url{ftp://ftp.cs.princeton.edu/pub/cs226/kdtree})
%
% 
   Repeat the question but with the 2d-tree implementation.}


\begin{table}[htbp]
\renewcommand{\arraystretch}{2}
%  \small
%\baselineskip 1.5\baselineskip
  \centering
  \caption{!Insert caption!}
        \label{tab:table1}
        \begin{tabular}{|r| c | c |}
          \multicolumn{3}{r}{calls to \texttt{nearest()} per second} \\
        \hline
         & \qquad \emph{brute force} \qquad & \qquad \emph{$2d$-tree} \qquad \\
        \hline
        \texttt{input100K.txt} & & \\\hline
        \texttt{input1M.txt} & & \\
        \hline
        \end{tabular}
\end{table}



\section{About This Solution}

Have you taken (part of) this course before:
\smallskip

Hours to complete assignment (optional):


% /******************************************************************************
%  *  After reading the course rules on collaboration policy, answer the
%  *  following short quiz. This counts for a portion of your grade.
%  *  Write down the answers in the space below.
%  *****************************************************************************/
 
\subsection{Known Bugs / Limitations.}
% /******************************************************************************
\explanation{Known bugs / limitations. For example, if your program prints
  out different representations of the same line segment when there
 are 5 or more points on a line segment, indicate that here.}
%  *  Known bugs / limitations.
%  *****************************************************************************/

\subsection{Help Received}
% /******************************************************************************
\explanation{
Describe whatever help (if any) that you received.
Don't include readings, lectures, and classes, but do
include any help from people (including course staff, lab TAs,
classmates, and friends) and attribute them by name.}
%  *****************************************************************************/


\subsection{Problem Encountered}
% /******************************************************************************
\explanation{
Describe any serious problems you encountered.                    }
%  *****************************************************************************/



\subsection{Comments}
% /******************************************************************************
\explanation{
List any other comments here. Feel free to provide any feedback   
on how much you learned from doing the assignment, and whether    
you enjoyed doing it.}
% *****************************************************************************/




\end{document}

%%%% Optional material

% Example plot
%    \begin{figure}[!ht]
%        \label{fig:plot1}
%        \centering
%        \includegraphics[width=0.6\textwidth]{chart.pdf}
%        \caption{This is an imported figure}
%    \end{figure}

% Example reference:    See Listing \hyperref[lst:random]{1}.

    \pagebreak
    \section*{Appendix I: Source code listings}
    Optional.

    \subsection*{Acknowledgement}

    % Environment for listing code
    \begin{lstlisting}[caption={This is a caption.},label={lst:array1}]
public class This is a class {

    // This is a comment
    public static double thisIsAFunction(int N) {
        QuickUnionUF uf = new QuickUnionUF(N);
        return something;
    }
    public static void main(String[] args) { 
        int T = 50;
    }
} 
    \end{lstlisting}


    \pagebreak
    \section*{Appendix II}
    Optional.

    \listoftables
    \listoffigures
    \lstlistoflistings
    \bibliographystyle{plain}
    \bibliography{s1.bib}

\end{document}

